\documentclass[10pt]{paper}
\usepackage{palatino}
\usepackage{hyperref}
\usepackage{listings}
\lstset{language=Python,
        stringstyle=\ttfamily,
        escapeinside={*@}{@*},
        showstringspaces=false}

 \title{Test Driven Development of Bib\TeX Author Strings}
\date{2017}
\begin{document}
\maketitle
\section*{Introduction}
Bib\TeX\footnote{\url{http://en.wikipedia.org/wiki/BibTeX}} is a
reference management system that is used with La\TeX to ease the use
of citations in documents. A Bib\TeX database is a text file
containing bibliographic information. A typical entry might look like:
\begin{verbatim}
@Article{hell:duality_tree_homo,
  author = 	 {Hell, P. and Ne\v{s}et\v{r}il, J. and Zhu, X. },
  title = 	 {Duality and Polynomial Testing of Tree Homomorphisms},
  journal = 	 {Transactions of the American Mathematical Society},
  year = 	 {1996},
  volume =	 {348},
  number =	 {4},
  pages =	 {1281-1297}
}
\end{verbatim}

For the purpose of sorting the entries in the bibliography it is
important to know the surname and forenames of an author. Bib\TeX
provides a convention where surname and forenames can be unambiguously
given. In general there are two ways of expressing a name.
\begin{verbatim}
   Surname, Forename1 Forename2
\end{verbatim}
or
\begin{verbatim}
 Forename1 Forename2 Surname
\end{verbatim}
The reason for this complication is that {\tt Ludwig van Beethoven}
has the surname {\tt van Beethoven}. So this can be expressed in
Bib\TeX either as 
\begin{verbatim}
van Beethoven, Ludwig
\end{verbatim}
or
\begin{verbatim}
Ludwig {van Beethoven}
\end{verbatim}

A list of names is separated by {\tt and}. 

\section*{Goal}
The goal of this lab is to write a python function {\tt
  extract\_authors} that takes a string of names separated by {\tt
  and}  and return a list of pairs or strings {\tt
  ('Surname','Forenames')}.
You will do this by test driven development. First you will develop a
function {\tt extract\_author} that takes a single author and returns a
single pair {\tt   ('Surname','Forenames')}.  You can use this
function to make {\tt extract\_authors}.

\section*{Lab Instructions}
As you develop your code you need to keep a code diary. In a separate text
file paste each test and the code that you write to pass the test. It
is important that you follow TDD strictly. Only write enough code to
pass the test and refactor when necessary. 

There is nothing to hand in with this lab. You will be orally marked
by one of the lab assistants. You will show code diary  that have produced,
and the lab assistants will ask you questions about the code. The lab
can be done in pairs, but both members of the group are required to
understand the code that you have written.  


\section*{Test Cases}
The following listing contains all the test cases that your code needs
to pass. You develop the code using the tests in the {\tt order}
given.   Note that you are expected to type in the test cases
yourself.

\begin{lstlisting}
  import bibtex
  import unittest
  class TestAuthorExtract(unittest.TestCase):
    """ Functions for testing the extraction of author names.

    Note that the rules for extracting bibtex author names is actually
    quite complicated.  As these test cases illustrate. I haven't
    implemented all the cases. """

    def setUp(self):
        self.simple_author_1 =   "Smith"
        self.simple_author_2 =   "Jones"
        self.author_1 = "John Smith"
        self.author_2 = "Bob Jones"
        self.author_3 = "Justin Kenneth Pearson"
        self.surname_first_1 = "Pearson, Justin Kenneth"
        self.surname_first_2 = "Van Hentenryck, Pascal"
        self.multiple_authors_1 = "Pearson, Justin and Jones, Bob"
    def test_author_1(self):
        #Test only surnames.
        (Surname,FirstNames) = bibtex.extract_author(self.simple_author_1)
        self.assertEqual( (Surname,FirstNames) ,  ('Smith','') )
        (Surname,FirstNames) = bibtex.extract_author(self.simple_author_2)
        self.assertEqual( (Surname,FirstNames) ,  ('Jones','') )
    def test_author_2(self):
        #Test simple firstname author.
        (Surname,First) = bibtex.extract_author(self.author_1)
        self.assertEqual( (Surname,First) , ("Smith","John"))
        (Surname,First) = bibtex.extract_author(self.author_2)
        self.assertEqual( (Surname,First) , ("Jones","Bob"))
    def test_author_3(self):
        (Surname,First) = bibtex.extract_author(self.author_3)
        self.assertEqual( (Surname,First) , ("Pearson","Justin Kenneth"))
    def test_surname_first(self):
        (Surname,First) = bibtex.extract_author(self.surname_first_1)
        self.assertEqual( (Surname,First) , ("Pearson","Justin Kenneth"))
        (Surname,First) = bibtex.extract_author(self.surname_first_2)
        self.assertEqual( (Surname,First) , ("Van Hentenryck","Pascal"))
    def test_multiple_authors(self):
        Authors = bibtex.extract_authors(self.multiple_authors_1)
        self.assertEqual (Authors[0] ,  ('Pearson','Justin'))
        self.assertEqual (Authors[1] , ('Jones', 'Bob') )
if __name__ == '__main__':
  unittest.main()
\end{lstlisting}


\section*{Hints}
There are a lot of helpful function in Python for string handling. You
should look at {\tt join},{\tt split} and {\tt strip}. Also some of
the string handling exercises that are in the pre-lab might be useful. 


%\section*{Dates}
%The lab can be marked on one of the following  occasions:

%TODO Fill in.
\end{document}

%%% Local Variables:
%%% mode: latex
%%% TeX-master: t
%%% End:
