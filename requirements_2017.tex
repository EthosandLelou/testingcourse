\documentclass[a4page]{article}
\usepackage{palatino}
\usepackage{fullpage}
\usepackage{hyperref}
\newcommand{\todo}[1]{{\tt #1}}
\title{Software Testing 2017}
\author{Justin Pearson}



 
\begin{document}
\maketitle


\section{Introduction}
The purpose of this document is to make clear to you how will be examined and
what is required of you in the course.  The course essentially consists of
four components:
\begin{enumerate}
\item A series of lectures on aspects of software testing
\item Group work on test design.
\item A lab on test driven development.
\item An exam.
\end{enumerate}
The exam is graded U,3,4,5. The group work and the project is pass or fail.

If you only learn one thing about testing during this course, then you will
fail, but if you learn the following idea ``You should have a reason for each
test. For each test you should be able to explain what you are trying to test
for.'' then you will at least understand why we study testing. When we examine
you and you produce tests, we expect you to be able to explain the reasons
behind each test or group of tests.


\section{Deliverables and requirements  2017}



\begin{enumerate}
\item A lab on  test driven development (TDD) for  BibTex.   This is
  done in groups of 2 or 3. It will be marked at the lab. There is no
  hand-in. 


\item Project work on test case design for some python library. This
  is done in groups of 4.
\item An exam.

  \end{enumerate}


  Information on the lab can be found either linked from the student portal or
  via
  \url{http://user.it.uu.se/~justin/Teaching/Testing/index.html}. Information
  on the exam will come later. It is safe to say that if I mention a concept
  on the slides then I might ask an exam question on it. It is also safe to
  say, that my slides do not contain all the information that you need. You
  will probably have to read the book to get enough information to pass the
  exam.

\subsection{Project Work}

The idea of the project is to pick some python library, and write some
test cases for it. You will have to write both black-box and white box
tests and tests that provide coverage of selected parts of the
code. The final deliverable will be a written document and a
presentation on the 30/11. 
  \begin{itemize}
  \item By  the start of  week  46 you must form a group. This will be
    handled by the student portal.
  \item In week 46 you will meet your assigned lab assistant, before the
    meeting you should pick a python library (for example look at the Python
    Standard Library or \url{https://pypi.python.org/}) you must agree
    with your lab assistant that you can work on the library. We have
    to decide if the library is too trivial or too complicated.
  \item You are to derive a number of test cases:
    \begin{itemize}
    \item  Black box testing of the API. You are to produce test cases
      that cover the API.
    \item  White box testing. In agreement with your lab assistant you
      are to pick some areas of code in your library to cover. Together with
      the python library
      \url{https://coverage.readthedocs.io/en/coverage-4.4.1/} you are
      to produce test cases that not only provide statement coverage
      but some sort of path coverage. You need to document and
      motivate your design of the test cases. You should meet your lab
      assistant in week 47.
    \item In week 48 or 49 you again meet with your lab assistant to
      discuss your draft of your  report and your test cases.
    \item On 30/11 there is a compulsory presentation, where you are
      too present your library and present some of your tests cases
      and the reasoning behind them. Your presentation is of work in
      progress. 
    \item Jan 8th 2018 is  the deadline for the final
      report. 
    \end{itemize}
  \end{itemize}

  \subsection*{Summary of meetings}
  \begin{itemize}
  \item Week 46  --- Discuss your choice of library.
  \item Week 47  --- Discuss your test cases and choice of code that you are
    going to provide test coverage.
  \item Week 48 or 49 --- Discuss the draft of your report.
  \end{itemize}
  \subsection*{Summary of classes of test cases}
  \begin{itemize}
  \item Black tests of the API. You should provide test cases that cover a
    large  (or all) of the API of the library.
  \item White box tests that provide coverage of part of the library.
  \end{itemize}
  When documenting your test cases you have document what your tests are
  designed to do.

  \subsection*{What should your report contain}
  \begin{itemize}
  \item Description of the python library with examples of use. This should be
    written for somebody who has not read the documentation.
  \item Documentation and all tests cases.
  \end{itemize}
\section{Grading criteria for written deliverables}
\begin{itemize}
\item Clarity of presentation. It is important that your document can
  be read without reference to the documentation of the library. This
  means that you will have to reexplain things already in the
  documentation. 
\item Quality of the test cases. What sort of coverage do their
  provide? Do they test enough functionality. 
\item Documentation of the test cases. These can be in comments in
  your code, for each test case of group of test cases you must
  provide reasons for the tests and document  what you are trying to
  test.
 
\end{itemize}


\end{document}

%%% Local Variables:
%%% mode: latex
%%% TeX-master: t
%%% End:
