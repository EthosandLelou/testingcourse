%\documentclass[handout]{beamer}
\documentclass{beamer}
\usetheme{default}
\usepackage{hyperref}
\usepackage{listings}
\title{Software Testing \\ Introduction}
\author{Justin Pearson}
\date{2019}
\setbeamertemplate{footline}[page number]
\setbeamertemplate{navigation symbols}{}
\begin{document}
\lstset{language=python}

\begin{frame}
  \maketitle
\end{frame}
\begin{frame}
  \frametitle{Outline}
  \begin{itemize}
  \item Lectures.
  \item Lab on Test driven development. Groups of 2.
  \item Project, groups of 4. Test case development for a python
    library of your choice.
  \item Exam
  \end{itemize}
\end{frame}
\begin{frame}
  \frametitle{Grading}
  \begin{itemize}
  \item The Exam is U,3,4,5
  \item Project work and the lab is pass/fail.
  \end{itemize}
\end{frame}
\begin{frame}
  \frametitle{Lab}
  \begin{itemize}
  \item Test driven development exercise. Using author name parsing in
    BibTex. 
  \end{itemize}
\end{frame}
\begin{frame}
  \frametitle{Project}
  \begin{itemize}
  \item White box and Black box testing for a python library of your choice.
  \item Presentation 4/12
  \item Written report deadline 2018-01-07.
  \item You must document what your tests are designed to do.
  \end{itemize}
\end{frame}
\begin{frame}
  \frametitle{Lecture slides and revision}
  \begin{itemize}
  \item   My lecture slides are rather sparse. You will not be able to pass
  the exam by simply looking at the slides. You will need to read the
  relevant sections of the book. I will put up chapter links on
  \url{http://user.it.uu.se/~justin/Teaching/Testing/index.html}. Please note
  that the chapter links are for the 2nd edition of the book. This is much
  improved over the first edition of the book. 
\item Some of the lab and project work will also prepare you for the
  exam. From your work on test design and documentation you will
  prepare yourself to answer some more reflective questions on the exam.
  \end{itemize}
\end{frame}
\begin{frame}
  \frametitle{What will you learn?}
  \begin{itemize}
  \item Software testing is an interesting mix of art and theory.
    \begin{itemize}
    \item Theory tells you that testing is impossible, but
    \item Testing does improve software quality.
    \end{itemize}
  \item I will give you the tools to develop tests in a more
    principled way. You will be given the theoretical tools to think
    about questions such as:
    \begin{itemize}
    \item What do my tests cover?
    \item What does coverage actually mean? 
    \end{itemize}
  \end{itemize}
\end{frame}
\end{document}

%%% Local Variables:
%%% mode: latex
%%% TeX-master: t
%%% End:
