\documentclass{beamer}

\begin{document}
\title{Software Testing}
\author{Justin Pearson}
\date{}
\begin{frame}
\maketitle
  
\end{frame}
\begin{frame}
  \frametitle{Some Questions}
  \begin{itemize}
  \item Does my software work?
  \item I've fixed a bug, does my software still work?
  \item I'm not really sure what this piece of code should do, I need
    a quick and easy way of expressing it.
  \end{itemize}
  One answer software testing.
\end{frame}
\begin{frame}
  \frametitle{Software Testing}
  \begin{itemize}
  \item 5 Credit advanced level course
  \item Period 2
  \item Examined by a mixture of exams; oral and written
    assignments;  and a small project.
  \item Entry requirements: Imperative and Object oriented
    programming, algorithms and data structures. %Compiler design is
%    recommended.  
  \end{itemize}
\end{frame}
\begin{frame}
  \frametitle{Course Goals}
  \begin{itemize}
  \item Learn key techniques  software testing, such as unit testing,
    test driven development, test coverage,  and test design.
  \item Presenting APIs, devising test cases 
  \end{itemize}
\end{frame}

\begin{frame}
  \frametitle{At the end of the course}
  Software testing is a large subject. There is a limit to what you
  can cover in a 5hp course, but 
  \begin{itemize}
  \item You will discover that software testing is a easy to implement
    techniques that improves software quality and should be in any
    programmers toolbox.
  \item I will give you the theoretical foundations so you can ask
    and investigate such questions as:
    \begin{itemize}
    \item Are all my requirements covered by my test cases?
    \item Have I tested all my code?
    \end{itemize}
  \end{itemize}
\end{frame}
%\begin{frame}
%  \frametitle{More notes on the requirements}
%  \begin{itemize}
%  \item You should be familiar with Unix command line.
%  \item The compiler course is not a prerequisite, but you should have
%    some idea what compilers do.
%  \item The Python core is written in C. You need to be able to
%    program in C (or are willing to learn).
%  \item Python is not a prerequisite, you can learn what you need
%    during the course.
%  \end{itemize}
%Again: This course is an intensive course and it will take the hours
%assigned. Start early, identify gaps in your knowledge. It is too late
%after Christmas.
%\end{frame}
%
\end{document}
